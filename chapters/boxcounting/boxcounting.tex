
\chapter{Box-counting Dimension}

\section{Introduction}

\section{Covers}

Let $F$ be a subset of $\RR^n$.
We define the following covers:
\begin{align*}
	\mcb_\delta(F)&:=\set{B=\set{B(x_j,\delta)}_{j=1}^{n_B}| F\subseteq \cup B}\\
	\mcc_\delta(F)&:=\set{C=\set{\bar{B}(x_j,\delta)}_{j=1}^{n_C} F\subseteq \cup C}\\
	\mcd_\delta(F)&:=\set{D=\set{D_j}_{j=1}^{n_D}| F\subseteq \cup D \,\,\wedge\,\, \diam(D_j)\leq\delta\,\, \forall\,\, D_j\in D}.
\end{align*}
We now define the following counts
\begin{align*}
	N^B_\delta(F)&:=\min\limits_{B\in\mcb_\delta(F)}|B|\\
	N^C_\delta(F)&:=\min\limits_{C\in\mcc_\delta(F)}|C|\\
	N^D_\delta(F)&:=\min\limits_{D\in\mcd_\delta(F)}|D|.
\end{align*}

\begin{thm}
	\[
		N^B_\delta(F)=N^D_\delta(F)
	\]
\end{thm}
\begin{proof}
	The proof is trivial and left as an excercise.
\end{proof}

\section{Box-counting Dimension}

\begin{definition}
	We say the \textit{lower box-counting dimension} of $F\subset\RR^n$ is
	\[
		\diml_B(F):=\liminf\limits_{\delta\rightarrow 0} \frac{\log N^B_\delta(F)}{-\log\delta}
	\]
	and the \textit{upper box-counting dimension} is
	\[
		\dimu_B(F):=\limsup\limits_{\delta\rightarrow 0} \frac{\log N^B_\delta(F)}{-\log\delta}.
	\]
	If $\diml_B(F)=\dimu_B(F)$, then we say the \textit{box-counting dimension} of $F$ is
	\[
		\dim_B(F):=\lim\limits_{\delta\rightarrow 0} \frac{\log N^B_\delta(F)}{-\log\delta}.
	\]
\end{definition}

\begin{thm}
	$N^C_\delta(F)$ yields the same box-counting dimension as $N^D_\delta(F)$.
\end{thm}
\begin{proof}
	The details are trivial and left as an exercise.
\end{proof}

\section{Examples}

\begin{lemma}\label{lems}
	Let $\set{x_n}_{n=1}^\infty$, be a monotone decreasing sequence in $[0,1]$ that tends to 0 and let $0 < \delta < 1/2$.
	Let $S=\set{x_n}_{n=1}^\infty\cup\set{0}$.
	Define
	\begin{align*}
		l_S(n)&:= x_n-x_{n+1}\\
		r_S(n)&:= x_{n-1}-x_n.
	\end{align*}
	Let $k\in\NN$ such that $l_S(k)\leq \delta < r_S(k)$, then
	\[
		k \leq N^C_\delta(S)\leq 2k.
	\]
	and
	\[
		\frac{\ln k}{-\ln r_S(k)}\leq \frac{\ln N^C_\delta}{-\ln\delta}\leq\frac{\ln (2k)}{-\ln l_S(k)}.
	\]
\end{lemma}
\begin{proof}
	The proof is obvious and thus is omitted here.
\end{proof}

\begin{example}
	Let $j\in\QQ^+$, then the box-counting dimension of $S_j=\set{n^{-j}}_{n=1}^\infty\cup\set{0}$ is
	\[
		\dim_B(S_j)=\frac{1}{j+1}.
	\]
\end{example}
\begin{proof}
	From the construction of $S_j$ we know that
	\[
		l_{S_j}(k) = \frac{1}{k^j}-\frac{1}{(k+1)^j} = \frac{(k+1)^j-k^j}{k^j(k+1)^j}
	\]
	and
	\[
		r_{S_j}(k) = \frac{1}{(k-1)^j}-\frac{1}{k^j} = \frac{k^j-(k-1)^j}{k^j(k-1)^j}.
	\]
	By Lemma\autoref{lems}, we know that
	\[
		\frac{\ln k}{-\ln r_{S_j}(k)}\leq \frac{\ln N^C_\delta}{-\ln\delta}\leq\frac{\ln (2k)}{-\ln l_{S_j}(k)}.
	\]
	Thus, the box-counting dimension is in between the limits of the far left and far right of this inequality.
	Consider $\ln k/(-\ln r_{S_j}(k))$.
	\[
		\frac{\ln k}{-\ln r_{S_j}(k)}=\frac{\ln k}{-\ln (k^j-(k-1)^j)+j\ln k +j\ln(k+1)}
	\]
	Consider $n^j-(n-1)^j$.
	\[
		n^j-(n-1)^j=n^{j-1}\paren{ n - (n-1)\paren{\frac{n-1}{n}}^{j-1} }=n^{j-1}\paren{ n - (n-1)\paren{1-\frac{1}{n}}^{j-1} }
	\]
	Thus,
	\[
		\frac{\ln k}{-\ln r_{S_j}(k)}=\frac{\ln k}{-\ln(k-(k-1)(1-1/k)^{j-1})+\ln k +j\ln(k+1)}
	\]
	which tends to $1/(j+1)$.
	Consider $\ln 2k/(-\ln l_{S_j}(k))$.
	\[
		\frac{\ln 2k}{-\ln l_{S_j}(k)}=\frac{\ln 2k}{-\ln ((k+1)^j-k^j)+j\ln k +j\ln(k-1)}
	\]
	Consider $(n+1)^j-n^j$.
	\[
		(n+1)^j-n^j=n^{j-1}\paren{ (n+1)\paren{\frac{n+1}{n}}^{j-1} - n }=n^{j-1}\paren{ (n+1)\paren{1+\frac{1}{n}}^{j-1} - n }
	\]
	Thus,
	\[
		\frac{\ln 2 +\ln k}{-\ln l_{S_j}(k)}=\frac{\ln 2+\ln k}{-\ln((k+1)(1+1/k)^{j-1}-k)+\ln k +j\ln(k+1)}
	\]
	which tends to $1/(j+1)$ and by Squeeze theorem, $\dim_B(S_j)$ is $1/(j+1)$.
\end{proof}

\begin{example}
	Let $S=\set{(n!)^{-1}}_{n=1}^\infty\cup\set{0}$, then
	\[
		\dim_B(S)=0.
	\]
\end{example}
\begin{proof}
	We begin by computing $l_S(n)$ and $r_S(n)$.
	\begin{align*}
		l_S(n) &= \frac{n}{(n+1)!}\\
		r_S(n) &= \frac{n-1}{n!}
	\end{align*}
	We then invoke Lemma\autoref{lems} to get
	\[
		\frac{\ln k}{-\ln (k-1)+\ln(k!)} \leq \frac{\ln N^C_\delta(S)}{-\ln\delta} \leq \frac{\ln 2+\ln k}{-\ln k +\ln ((k+1)!)}.
	\]
	Since the limit of both the far right and the far left of the previous inequality is zero, $\dim_B(S)=0$.
\end{proof}

\begin{example}
	Let $C$ denote the middle-third Cantor set.
	\[
		\dim_B(C)=\frac{\ln2}{\ln3}
	\]
\end{example}
\begin{proof}
	The Cantor set is wierd.
\end{proof}

