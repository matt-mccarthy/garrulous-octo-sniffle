
\chapter{Hausdorff Dimension}

After exploring the deficiencies of the box-counting dimension, we need a better way to determine the dimension of the set.
In order to do so, we must somehow account for the ``size'' of each set in our cover.
Ideally, we want to encompass the properties of both the counting measure and Lesbesque measure, and thus we introduce the Hausdorff measure.

\section{The Hausdorff Measure}

We begin by defining a $\delta$ cover.
\begin{definition}
	Let $F$ be a subset of a metric space $(X,d)$.
	Then a \textit{$\delta$ cover of $F$} is $\set{U_i}_{i\in I}$ where $I$ is countable, and $\diam U_i \leq\delta$ for all $i\in I$.
	We denote the family of all $\delta$-covers of $F$ as $C_\delta(F)$.
\end{definition}

We define the $s$-dimensional Hausdorff-$\delta$ thingy as follows.

\begin{definition}
	Let $F\subseteq\RR^n$, and $s,\delta>0$ then the \textit{$s$-dimensional Hausdorff-$\delta$ thingy of $F$} is defined as
	\[
		\hm^s_\delta=\inf\set{\sum_{i\in I} {(\diam U_i)}^s | \set{U_i}_{i\in I} \text{ a $\delta$-cover of $F$}}.
	\]
\end{definition}
Note that as $\delta$ tends to zero, the collection of covers over which we take the infinimum becomes smaller.
This tells us that at the very least, $\hm^s_\delta(F)$ is non-increasing and thus will approach a limit as $\delta$ tends to zero.
In turn, this leads us to the definition of the $s$-dimensional Hausdorff measure.

\begin{definition}
	Let $F\subseteq\RR^n$ and $s>0$, then the \textit{$s$-dimensional Hausdorff measure of $F$} is defined as
	\[
		\hm^s(F) = \lim_{\delta\rightarrow0} \hm^s_\delta,
	\]
\end{definition}

\begin{thm}
	For $s>0$, $\hm^s$ is a measure on the Borel $\sigma$-algebra of $\RR^n$.
\end{thm}

\begin{proof}
	Let $z\in\RR^n$ and $X, Y\subseteq\RR^n$.
	Define $D:\RR^n\times\mathcal{P}(\RR^n)\rightarrow\RR^+$ by
	\[
		D(z,X) := \inf_{x\in X} |z-x|
	\]
	and $D:{\mathcal{P}(\RR^n)}^2\rightarrow\RR$ by
	\[
		D(X,Y) := \inf_{x\in X,\, y\in Y} |x-y|.
	\]

	To show $\hm^s$ is a measure, we need to show that $\hm^s(\emptyset)=0$ and for $A,B\subseteq\RR^n$, with $A\cap B=\emptyset$, $\hm^s(A\cup B)=\hm^s(A)+\hm^s(B)$.
	We begin by showing that $\hm^s(\emptyset)=0$.
	Note that the minimal $\delta$-cover of $\emptyset$ is an empty cover, thus the sum of its diameters is zero.
	Ergo $\hm^s(\emptyset)=\lim\limits_{\delta\rightarrow 0} \hm^s_\delta(\emptyset)=\lim\limits_{\delta\rightarrow 0} 0 =0$.

	Let $\set{F_i}_{i\in I}$ be a countable collection of subsets of $\RR^n$.
	For each $i\in I$, let $\mathcal{U}\in C_\delta(F_i)$.
	Then $\mathcal{W}=\cup_{i\in I} \mathcal{U_i}\in C_\delta(\cup_{i\in I} F_i)$ for any combination of covers of the individual sets.
	Thus,
	\[
		\hm^s_\delta
		= \inf_{\mathcal{W}\in C_\delta(\cup F_i)} \sum_{W\in\mathcal{W}}{(\diam W)}^s
		\leq \sum_{i\in I} \inf_{\mathcal{U}\in C_\delta(F_i)}\sum_{U\in\mathcal{U}} {(\diam U)}^s
		=\sum_{i\in I} \hm^s_\delta(F_i)
	\]
	and by limit laws, $\hm^s(\cup F_i)\leq \sum_{i\in I} \hm^s(F_i)$.
	% TODO: Reverse the inequality
\end{proof}

It should be noted that if $s=0$, $\hm^s$ is exactly the counting measure, and that if $s$ is a natural number then $\hm^s$ is the Lesbesque measure up to a constant multiple.

\begin{thm}
	$\hm^0(F)=\card F$
\end{thm}
\begin{proof}
	Suppose $F$ is countable and let $\delta>0$.
	Denote $\mathcal{F}=\set{\set{x}}_{x\in F}$, then $\mathcal{F}$ is the minimal $\delta$-cover of $F$.
	Thus,
	\[
		\hm^0_\delta(F)=\sum_{x\in F} {(\diam\set{x})}^0=\card F
	\]
	and $\hm(F)=|F|$.

	If $F$ is uncountable then, there exists a countably infinite subset of $F$, which we denote $F_c$.
	Since $\hm^0$ is a measure, $\hm^0(F_c)=\infty\leq\hm^0(F)$.
	Thus, $\hm^0(F)=\infty$.
\end{proof}

\subsection{Properties of the Hausdorff Measure}

\begin{thm}\label{hm-holder}
	Let $F\subset\RR^n$ and let $f: F\rightarrow\RR^m$ be satisfy the H\"older condition
	\[
		|f(x)-f(y)| \leq c|x-y|^\alpha.
	\]
	Then for any s,
	\[
		\hm^{s/\alpha}(f(F))\leqq c^{s/\alpha}H^s(F).
	\]
\end{thm}
\begin{proof}
	Let $\delta>0$ be given.
	Let $\mathcal{U}\in C_\delta(F)$.
	Then by H\"older property,
	\[
		\diam f(F\cap U)\leq c{(\diam F\cap U)}^\alpha \leq c{(\diam U)}^\alpha\leq c\delta^\alpha
	\]
	for all $U\in\mathcal{U}$.
	Thus, $\set{f(F\cap U)}_{U\in\mathcal U}\in C_{c\delta^\alpha}(f(F))$.
	Ergo,
	\[
		\sum_{U\in\mathcal{U}} {(\diam f(F\cap U))}^{s/\alpha}\leq c^{s/\alpha}\sum_{U\in\mathcal{U}}{(\diam U)}^s.
	\]
	And thus,
	\[
		\hm^{s/\alpha}_\delta (f(F)) \leq c^{s/\alpha}\hm^s_\delta F.
	\]
	Taking the limit as $\delta\rightarrow 0$, yields $\hm^{s/\alpha}(f(F))\leq c^{s/\alpha}\hm^s(F)$.
\end{proof}

\begin{thm}\label{hm-sim}
	Let $f:\RR^n\rightarrow\RR^n$ be a similarity transform with scale constant $\lambda$.
	Then $\hm^s(f(F))=\lambda^s\hm^s(F)$.
\end{thm}
\begin{proof}
	Since $f$ is a similarity transform with scale constant $\lambda$,
	\[
		|f(x)-f(y)|=\lambda|x-y|
	\]
	which is the H\"older condition with $\alpha=1$ and $c=\lambda$.
	Thus,
	\[
		\hm^s(f(F))\leq \lambda^s\hm^s(F).
	\]
	Since $f$ is a similarity, $f$ is invertible and
	\[
		|f^{-1}(x)-f^{-1}(y)|=\lambda^{-1}|x-y|
	\]
	which is the H\"older condition with $\alpha=1$ and $c=\lambda^{-1}$.
	Thus,
	\[
		\hm^s(F)=\hm^s(f^{-1}(f(F)))\leq\lambda^{-s}\hm^s(f(F))
	\]
	and
	\[
		\hm^s(f(F))=\lambda^s\hm^s(F).
	\]
\end{proof}

\begin{proposition}
	If we only consider $\delta$-covers consisting of closed sets when evaluating $\hm^s$ then the value remains unaltered.
\end{proposition}

\begin{proof}
	Let $F\subseteq\RR^n$ and let $\mathcal{U}\in C_\delta(F)$.
	Let $U\in\mathcal{U}$.
	We know $\diam U\leq\diam \bar{U}$, the closure of $U$.
	Let $\varepsilon > 0$ be given.
	Let $x,y\in\bar{U}$.
	Since $x,y$ are in the closure of $U$, then there exist sequences, $\set{x_n}_{n\in\NN}$ and $\set{y_n}_{n\in\NN}$ in $U$ such that $x_n\rightarrow x$ and $y_n\rightarrow y$.
	Ergo, by convergence, there exists an $N_x\in\NN$ such that for all $n\geq N_x$, $|x_n-x|<\varepsilon/2$.
	Similarly, we can find an $N_y$ such that $|y_n-y|<\varepsilon/2$ for all $n\geq N_y$.
	Take $N=\max\set{N_x,N_y}$.
	Then
	\[
		|x-y|\leq |x-x_N|+|x_N-y|\leq |x-x_N|+|x_N-y_N|+|y_N-y|\leq |x_N-y_N|+\varepsilon
	\]
	and $|x-y|\leq |x_N-y_N|$ since this holds for all $\varepsilon >0$.
	Taking supremums yields,
	\[
		\diam \bar{U} =\sup_{x,y\in\bar{U}} |x-y|\leq \sup_{a,b\in U} |a-b|=\diam U
	\]
	and $\diam\bar{U}=\diam U$.
	Therefore,
	\[
		\sum_{U\in\mathcal{U}}{(\diam U)}^s=\sum_{U\in\mathcal{U}}{(\diam\bar{U})}^s
	\]
	for all $\mathcal{U}\in C_\delta F$.
	Ergo, $\hm^s_\delta(F)$ remains unchanged and thus $\hm^s(F)$ remains unchanged if we only consider closed covers.
\end{proof}

\subsection{Examples of the Hausdorff Measure}

\begin{example}
	Let $F=[0,1]$, then: if $s < 1$, $\hm^s(F)=\infty$; if $s=1$, $0 <\hm^s(F) <\infty$; and if $s>1$, $\hm^s(F)=0$.
\end{example}
\begin{proof}
	Consider the sequence $1/n$.
	Since $\hm^s(F)=\lim\limits_{\delta\rightarrow0}\hm^s_\delta(F)$ exists,
	any subsequence of $\set{\hm^s_\delta(F)}_{\delta>0}$ converges to the same limit.
	Ergo, we will deal with $\set{\hm^s_{1/n}(F)}_{n=1}^\infty$ to simplify the proof.
	We begin by noting that the minimal $1/n$ cover of $F$ is $\set{i/n,(i+1)/n}_{i=0}^{n-1}$.
	Thus,
	\[
		\hm^s_{1/n}=\sum_{i=0}^{n-1}\frac{1}{n^s}=\frac{n}{n^s}=n^{1-s}.
	\]
	If $s < 1$, then $\hm^s_{1/n}\rightarrow\infty$.
	If $s > 1$, then $\hm^s_{1/n}\rightarrow 0$.
	However, if $s=1$, $\hm^s_{1/n}=n^{1-1}=1\rightarrow 1$.
\end{proof}

This critical value of $s=1$ is called the Hausdorff dimension of $F$.

\section{The Hausdorff Dimension}

\begin{definition}
	Let $F\subseteq\RR^n$, then the \textit{Hausdorff dimension of $F$} is defined as
	\[
		\dim_H F = \inf\set{s\geq 0 | \hm^s(F) = 0} = \sup\set{s\geq 0 | \hm^s(F)=\infty}.
	\]
\end{definition}

\section{Properties}

Recall the desired properties of any definition of dimension that we enumerated in chapter one.

\begin{thm}
	The Hausdorff dimension satisfies all of the following properties:
	\begin{enumerate}
		\item if $E\subseteq F$, then $\dim_H E\leq\dim_H F$;
		\item if $\set{S_i}_{i\in I}$ is a countable collection of sets, then $\dim_H \bigcup_{i\in I} S_i = \sup\limits_{i\in I}\dim_H S_i$;
		\item when $F\subseteq \RR^n$, $0\leq \dim_H F\leq n$;
		\item any open set $O\subseteq \RR^n$ should have $\dim_H O = n$;
		\item if $S$ is countable, then $\dim_H S=0$.
	\end{enumerate}
\end{thm}

\begin{proof}[Proof (1)]
	Suppose $E\subseteq F\subseteq\RR^n$.
	Since $\hm^s$ is a measure and $E\subseteq F$, then
	\[
		\hm^{\dim_H E} (E) \leq \hm^{\dim_H E} (F).
	\]
	Since $\hm^{\dim_H E}(E)$ is finite and $\hm^{\dim_H E}(F)$ may not be finite, $\dim_H E \leq \dim_H F$.
\end{proof}

\begin{proof}[Proof (2)]
	Let $\set{F_i}_{i\in I}$, be a countable collection of sets.
	By (1), we know that $\sup\limits_{i\in I}\, \dim_H F_i \leq\dim_H \cup_{i\in I} F_i$.
	Let $s > \sup\limits_{i\in I}\, \dim_H F_i$.
	Since $\hm^s$ is a measure, we know
	\[
		0\leq \hm^s (\cup_{i\in I} F_i) \leq \sum_{i\in I} \hm^s(F_i).
	\]
	However, since $s > \sup\limits_{i\in I}\, \dim_H F_i \geq \dim_H F_i$ for all $i\in I$ we have $\hm^s(F_i)=0$ for all $i\in I$ and $\hm^s (\cup_{i\in I} F_i) = 0$.
	Thus
	\[
		\dim_H \cup_{i\in I} F_i = \inf\set{s\geq 0 | \hm^s(\cup_{i\in I} F_i) = 0} =\sup\limits_{i\in I}\, \dim_H F_i
	\]
	and the Hausdorff dimension is stable.
\end{proof}

\begin{lemma}
	In $\RR^n$, $\dim_H B(x,r) = n$.
\end{lemma}
\begin{proof}
	% TODO: Prove it.
\end{proof}

\begin{proof}[Proof (3)]
	Let $F\subseteq\RR^n$.
	We know that we can cover $F$ with countably many open balls ${B_i}_{i\in I}$, each of dimension $n$.
	Thus $F\subseteq \cup_{i\in I} B_i$ and $\dim_H F \leq \dim_H \cup_{i\in I} B_i$ and by (4) we have
	\[
		\dim_H F \leq \sup\limits_{i\in I} \dim_H B_i = n.
	\]
	Since $\hm^s$ is a measure, we have $0\leq \dim_H F$.
\end{proof}

\begin{proof}[Proof (4)]
	Let $O\subseteq\RR^n$ be open.
	We know that $\dim_H O\leq n$ by (3).
	Since $O$ is open for each $x\in O$ there exists an $r>0$ such that $B(x,r)\subseteq O$.
	Pick any $x\in O$ and take any $r$ such that $B(x,r)\subseteq O$.
	Then $n=\dim_H B(x,r) \leq \dim_H O \leq n$ and $\dim_H O = n$.
\end{proof}

\begin{lemma}
	Let $x\in\RR^n$, then $\dim_H \set{x} = 0$.
\end{lemma}
\begin{proof}
	Note that the minimal $\delta$ cover of $\set{x}$ is $\set{x}$.
	Let $s > 0$, then
	\[
		\hm^s_\delta(\set{x})=\diam \set{x}^s = 0^s = 0
	\]
	and $\hm^s(\set{x}) = 0$.
	Thus,
	\[
		\dim_H \set{x} = \inf{s\geq 0| \hm^s(\set{x}) = 0} = 0.
	\]
\end{proof}

\begin{proof}[Proof (5)]
	Let $S\subset\RR^n$ be countable.
	Then for some $I\subseteq\NN$, $S=\set{s_i}_{i\in I}$.
	Thus,
	\[
		\dim_H S = \sup\limits_{i\in I} \dim_H s_i = 0
	\]
	by (4).
\end{proof}

\subsection{Remark on Stability}

Consider the set $S=\set{1/n}_{n\in\NN}$.
By property (5), we know that $\dim_H S=0$.
However, this raises the question as to what the Hausdorff dimension does differently than the box-counting dimension.
Suppose we're using the box-counting dimension with $N^D_\delta$.
In the cover of the set we used, when $\delta$ was between $1/k^2$ and $1/{(k-1)}^2$ we had to cover the first $k$ terms with sets of diameter no greater than $\delta$.
However, we could cover those $k$ terms, with singletons of the form $\set{1/k}$ or with balls of radius ${1/k^2}$ without affecting our count.

When we take the Hausdorff dimension of $S$, we still need to cover those $k$ terms, however we are forced by the infinimum in $\hm^s_\delta$ to take minimal cover, or in other words, cover those $k$ terms with singletons.
Since singletons have diameter zero, when $s>0$, they contribute absolutely nothing to our measure.
Since these are the majority of the sets in our cover, only a constant number of sets actually contribute any length to our measure and as we make $\delta$ approach zero, the contribution of these sets goes to zero.
Thus forcing the Hausdorff dimension of $S$ to be zero, since it cannot be positive or negative.

\subsection{Other Properties}

Like the box counting dimension, the Hausdorff dimension also has a nice property involving the H\"older condition.

\begin{thm}\label{hd-holder}
	Let $F\subset\RR^n$ and $f:F\rightarrow\RR^n$ satisfy the H\"older condition
	\[
		|f(x)-f(y)|\leq c{|x-y}^\alpha.
	\]
	Then $\dim_H f(F) \leq 1/\alpha \dim_H F$.
\end{thm}

\begin{proof}
	Suppose $s>\dim_H F$.
	Then, by Theorem\autoref{hm-holder}
	\[
		\hm^{s/\alpha}(f(F))\leq c^{s/\alpha}\hm^s(F)=0.
	\]
	Thus,
	\[
		\dim_H f(F) \leq \frac{s}{\alpha}
	\]
	for all $s > \dim_H$ which implies
	\[
		\dim_H f(F) \leq \frac{1}{\alpha}\dim_H F.
	\]
\end{proof}

\begin{corollary}
	If $f$ is bi-Lipschitz, then $\dim_H f(F)=\dim_H F$.
\end{corollary}

\begin{proof}
	Since $f$ is bi-Lipschitz, then $\dim_H f(F)\leq \dim_H F$, $f$ is invertible, and $f^{-1}$ is bi-Lipschitz.
	Since $f^{-1}$ is bi-Lipschitz, then $\dim_H F \leq \dim_H f(F)$ and $\dim_H f(F)\leq \dim_H F$.
\end{proof}

We also have a theorem that relates the Hausdorff dimension to the Box-counting dimension.

\begin{thm}
	Let $F\subset\RR^n$ then
	\[
		\dim_H F \leq \diml_B F.
	\]
\end{thm}
\begin{proof}
	Suppose that for some $s\geq 0$, $1<\hm^s(F)=\lim\limits_{\delta\rightarrow 0}\hm^s_\delta(F)$.
	Let $\mathcal{U}\in C_\delta(F)$, then
	\[
		1 < \sum_{U\in\mathcal{U}}{(\diam U)}^s
		\leq\sum_{U\in\mathcal{U}}\delta^s
		=\delta^s (\card \mathcal{U}).
	\]
	If we take the infinimum of both sides we get,
	\[
		1 < \hm^s_\delta(F)
		=\inf_{\mathcal{U}\in C_\delta(F)}\sum_{U\in\mathcal{U}} {(\diam{U})}^s
		\leq \delta^s\inf_{\mathcal{U}\in C_\delta(F)}\card{\mathcal{U}}
		=\delta^s N_\delta^D(F).
	\]
	Taking logs yields,
	\[
		0 < \ln N_\delta^D(F)+s\ln \delta
	\]
	which is equivalent to
	\[
		s\leq \frac{\ln N^D_\delta(F)}{-\ln\delta}.
	\]
	Thus, $\dim_H F\leq \diml_B F$.
\end{proof}

\begin{lemma}\label{hd-lem3}
	Let $F\subset\RR$ with $\hm^1(F) = 0$.
	Then $F^C$ is dense in $\RR$
\end{lemma}

\begin{proof}

\end{proof}

\begin{thm}
	Let $F\subset\RR^n$ with $\dim_H F < 1$, then $F$ is totally disconnected.
\end{thm}

\begin{proof}
	Let $x,y\in F$ with $x\neq y$.
	Define $f:\RR^n\rightarrow\RR^+$ by $f(z)=|z-x|$.
	Thus, by reverse triangle inequality,
	\[
		|f(z)-f(w)|=||z-x|-|w-x||\leq|z-x-w+x|=|z-w|
	\]
	and $f$ is Lipschitz.
	Ergo by Theorem\autoref{hd-holder}, $\dim_H f(F) \leq \dim_H F < 1$.
	Ergo, $\hm^1(f(F))=0$ and by Lemma\autoref{hd-lem3}, ${f(F)}^C$ is dense in $\RR$.
	Thus, we can choose an $r\notin f(F)$ such that $0<r<f(y)$.
	Thus, $F$ can be written as
	\[
		F=\set{z\in F| |z-x|< r}\cup\set{z\in F| |z-x| >r}.
	\]
	Note that both of those sets are disjoint and open.
	Furthermore, we can find these for any $x,y\in F$, thus $F$ must be totally disconnected.
\end{proof}

\subsection{Examples of the Hausdorff Dimension}

\begin{example}
	Let $F\subset\RR$ and let $f(x)=x^2$.
	Show that $\dim_H F = \dim_H f(F)$.
\end{example}

\begin{proof}
	Suppose $F$ is bounded, then $f$ is Lipschitz on $F$.
	Thus,
	\[
		\dim_H f(F) \leq \dim_H F.
	\]
	Let $F^+=F\cap\RR^+$ and $F^-=F\cap\RR^-$.
	Then $f:F^+\rightarrow f(F^+)$ and $f:F^-\rightarrow f(F^-)$ are invertible, and defined as $f^{-1}_+(x)=\sqrt{x}$ and $f^{-1}_-(x)=-\sqrt{x}$ respectively.
	Choose some $\varepsilon>0$ such that $F\setminus(-\varepsilon,\varepsilon)\neq\emptyset$.
	We know that $F^+\setminus[0,\varepsilon)$ or $F^-\setminus[0,\varepsilon)$ is nonempty. %chktex 9
	Without loss of generality assume $F^+\setminus[0,\varepsilon)\neq\emptyset$. %chktex 9
	Since $f^{-1}_+$ is continuously differentiable on $f(F^+\setminus[0,\varepsilon))$, $f^{-1}_+$ is Lipschitz. %chktex 9
	Ergo,
	\[
		\dim_H F^+\setminus[0,\delta) \leq \dim_H f(F^+)\leq \dim_H f(F) % chktex 9
	\]
	for all $0 < \delta \leq \varepsilon$.
	Taking the limit as $\delta\rightarrow 0$ yields,
	\[
		\dim_H F^+ \leq \dim_H f(F).
	\]
	A similar argument yields
	\[
		\dim_H F^- \leq \dim_H f(F).
	\]
	Thus by countable stability,
	\[
		\dim_H F = \max \set{\dim_H F^+, \dim_H F^-}\leq \dim_H f(F)
	\]
	and $\dim_H f(F)=\dim_H F$ when $F$ is bounded.

	Suppose $F$ is unbounded, then $F$ can be covered by countably many open balls of radius $r>0$ with centers $x_i$ in $F$.
	Denote, $F_i=B(x_i,r)\cap F$.
	Note that $F=\cup_{i\in\NN} F_i$ and $f(F)=\cup_{i\in\NN} f(F_i)$.
	Since $F_i$ is bounded, by the previous part, we know that $\dim_H F_i = \dim_H f(F_i)$ for each $i\in\NN$.
	Thus, by countable stability,
	\[
		\dim_H F = \sup\limits_{i\in\NN} \dim_H F_i = \sup\limits_{i\in\NN} \dim_H f(F_i)=\dim_H f(F).
	\]
\end{proof}

\begin{example}
	Let $f:[0,1]\rightarrow\RR$ be Lipschitz.
	Let $\gra f=\set{(x,f(x))|x\in[0,1]}$, then $\dim_H \gra f =1$.
\end{example}
\begin{proof}
	Let $g:\gra f\rightarrow [0,1]$ be defined by $g(x,f(x))=x$.
	We want to show that $g$ is bi-Lipschitz.
	Note that $|g(x,f(x))-g(y,f(y))|=|x-y|$ and that since $f$ is Lipschitz, $|f(x)-f(y)|\leq M|x-y|$.
	Consider $\sqrt{{(x-y)}^2+{(f(x)-f(y))}^2}$.
	\[
		\sqrt{{(x-y)}^2+{(f(x)-f(y))}^2}
		\leq \sqrt{{(x-y)}^2+M^2{(x-y)}^2}
		=\sqrt{M^2+1}|x-y|
	\]
	Thus, $\paren{\sqrt{{(x-y)}^2+{(f(x)-f(y))}^2}}/\sqrt{M^2+1}\leq|x-y|$.
	Moreover,
	\[
		|x-y|=\sqrt{{(x-y)}^2}\leq\sqrt{{(x-y)}^2+{(f(x)-f(y))}^2}.
	\]
	Ergo,
	\[
		\frac{1}{\sqrt{M^2+1}}\sqrt{{(x-y)}^2+{(f(x)-f(y))}^2}
		\leq |g(x,f(x))-g(y,f(y))|
		\leq \sqrt{{(x-y)}^2+{(f(x)-f(y))}^2}
	\]
	and $g$ is bi-Lipschitz.
	Therefore,
	\[
		\dim_H \gra f = \dim_H g(\gra f) = \dim_H [0,1] = 1.
	\]
\end{proof}

\begin{example}
	Let $F$ be the subset of $[0,1]$ whose elements have no instance of the digit five in their decimal expansion.
	Then $\dim_H F=\ln 9/\ln 10$.
\end{example}
\begin{proof}
	Define $F_i=[i/10,(i+1)/10]\cap F$.
	Then, $F=\cup_{i=0}^9 F_i$, the collection of $F_i$'s is mutually disjoint, $F_5=\emptyset$, and all other $F_i$'s are similar to $F$ with $\lambda=1/10$.
	Thus,
	\[
		\hm^s(F)=\sum_{i=0}^9 \hm^s(F_i)=\frac{9}{10^s}\hm^s(F).
	\]
	Assuming $0 < \hm^s(F) < \infty$ at the critical $s$, then we have
	\[
		10^s=9
	\]
	and
	\[
		s=\frac{\ln9}{\ln10}.
	\]
	Thus $\dim_H F=\ln 9 /\ln 10$.
\end{proof}
