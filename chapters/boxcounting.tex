
\chapter{Box-counting Dimension}

\section{Introduction}

\section{Covers}

Let $F$ be a subset of $\RR^n$.
We define the following covers:
\begin{align*}
	\mcb_\delta(F)&:=\set{B=\set{B(x_j,\delta/2)}_{j=1}^{n_B}| F\subseteq \cup B}\\
	\mcd_\delta(F)&:=\set{D=\set{D_j}_{j=1}^{n_D}| F\subseteq \cup D \,\,\wedge\,\, \diam(D_j)\leq\delta\,\, \forall\,\, D_j\in D}.
\end{align*}
We now define the following counts
\begin{align*}
	N^B_\delta(F)&:=\min\limits_{B\in\mcb_\delta(F)}\card(B)\\
	N^D_\delta(F)&:=\min\limits_{D\in\mcd_\delta(F)}\card(D).
\end{align*}

\begin{thm}
	\[
		N^B_\delta(F)=N^D_\delta(F)
	\]
\end{thm}
\begin{proof}
	Let $B$ be a cover in $\mcb_\delta(F)$ such that $\card(B)=N^B_\delta(F)$.
	Likewise let $D$ be a cover in $\mcd_\delta(F)$ such that $\card(D)=N^D_\delta(F)$.
	Since each set in $B$ has diameter $\delta$ $B\in\mcd_\delta(F)$, and by minimality, $N^D_\delta(F)\leq N^B_\delta(F)$.
	Since each set in $D$ has diameter no more than $\delta$, we can encapsulate each set in $D$ with a single ball of radius $\delta/2$, call this new cover $B(D)$.
	By construction, $D(B)\in\mcb_\delta(F)$ and $\card(D(B))=N^D_\delta(F)$.
	Again, by minimality, $N^B_\delta(F)\leq N^D_\delta(F)$ and $N^B_\delta(F) = N^D_\delta(F)$.
\end{proof}

\section{Box-counting Dimension}

\begin{definition}
	We say the \textit{lower box-counting dimension} of $F\subset\RR^n$ is
	\[
		\diml_B(F):=\liminf\limits_{\delta\rightarrow 0} \frac{\log N^B_\delta(F)}{-\log\delta}
	\]
	and the \textit{upper box-counting dimension} is
	\[
		\dimu_B(F):=\limsup\limits_{\delta\rightarrow 0} \frac{\log N^B_\delta(F)}{-\log\delta}.
	\]
	If $\diml_B(F)=\dimu_B(F)$, then we say the \textit{box-counting dimension} of $F$ is
	\[
		\dim_B(F):=\lim\limits_{\delta\rightarrow 0} \frac{\log N^B_\delta(F)}{-\log\delta}.
	\]
\end{definition}

\begin{thm}
	Let $f:\RR^n\rightarrow\RR^n$ be Lipschitz on $F\subset\RR^n$.
	Then $\diml_B(f(F))\leq\diml_B(F)$ and $\dimu_B(f(F))\leq\dimu_B(F)$.
	If $f$ is bi-Lipschitz on $F$, then $\diml_B(f(F))=\diml_B(F)$ and $\dimu_B(f(F))=\dimu_B(F)$.
\end{thm}
\begin{proof}
	Let $\delta > 0$ be given.
	Let $\set{U_i}_{i=1}^k$ be a minimal $\delta$-cover of $F$.
	Thus, $\set{U_i\cap F}$ is also a minimal $\delta$-cover of $F$.
	Note that for all $x,y\in U_i\cap F$, $|x-y|\leq \delta$.
	We also know that $\set{f(U_i\cap F)}$ is a $c\delta$-cover since $|f(x)-f(y)|\leq c|x-y|\leq c\delta$.
	However, this cover is not necessarily minimal and thus
	\[
		N_{c\delta}(f(F))\leq N_\delta(F).
	\]
	Ergo,
	\[
		\frac{\ln N_{c\delta}(f(F))}{-\ln(c\delta)+\ln\delta}\leq\frac{\ln N_\delta(F)}{-\ln\delta}
	\]
	and $\diml_B(f(F))\leq\diml_B(F)$ and $\dimu_B(f(F))\leq\dimu_B(F)$.
	
	If $f$ is bi-Lipschitz, then $f^{-1}$ is Lipschitz on $f^{-1}(F)$.
	Thus $\diml_B(f^{-1}(f(F)))\leq\diml_B(f(F))$ and $\dimu_B(f^{-1}(f(F)))\leq\dimu_B(f(F))$.
	Therefore, $\diml_B(f(F))=\diml_B(F)$ and $\dimu_B(f(F))=\dimu_B(F)$.
\end{proof}

\begin{thm}
	Suppose $f$ satisfies the H\"{o}lder condition, that is for some $c,\alpha>0$
	\[
		|f(x)-f(y)|\leq c|x-y|^\alpha.
	\]
	Then $\diml_B(f(F))\leq(1/\alpha)\diml_B(F)$ and $\dimu_B(f(F))\leq(1/\alpha)\dimu_B(F)$.
\end{thm}
\begin{proof}
	Let $\delta > 0$ be given.
	Let $\set{U_i}_{i=1}^k$ be a minimal $\delta$-cover of $F$.
	Thus, $\set{U_i\cap F}$ is also a minimal $\delta$-cover of $F$.
	Note that for all $x,y\in U_i\cap F$, $|x-y|\leq \delta$.
	We also know that $\set{f(U_i\cap F)}$ is a $c\delta^\alpha$-cover since $|f(x)-f(y)|\leq c|x-y|^\alpha\leq c\delta^\alpha$.
	However, this cover is not necessarily minimal and thus
	\[
		N_{c\delta^\alpha}(f(F))\leq N_\delta(F).
	\]
	Ergo,
	\[
		\frac{\ln N_{c\delta^\alpha}(f(F))}{-\ln(c\delta^\alpha)+\ln\delta}\leq\frac{\ln N_\delta(F)}{-\alpha\ln\delta}
	\]
	and $\diml_B(f(F))\leq(1/\alpha)\diml_B(F)$ and $\dimu_B(f(F))\leq(1/\alpha)\dimu_B(F)$.
\end{proof}

\section{Examples}

\begin{example}
	Let $C$ denote the middle-$\lambda$ Cantor set for $0<\lambda < 1$.
	\[
		\dim_B(C)=\frac{\ln2}{\ln\paren{\frac{2}{1-\lambda}}}
	\]
\end{example}
\begin{proof}
	To begin, we must determine the length of each interval of $C_n$ after we remove the middle $1/\lambda$ from each of them.
	Assume we are removing the middle $\lambda$ of the interval $[0,a]$ where $a$ is a positive real number.
	Since we removed $a\lambda$ from $[0,a]$, we have exactly $a-a\lambda=a(1-\lambda)$ length remaining.
	Since we removed the middle $a/\lambda$, we equally distribute the remaining length into two subintervals of length $a(1-\lambda)/2$.
	
	Let $l_n$ represent the length of each interval in $C_n$ and define $l_0=1$.
	Thus, $l_1=(1-\lambda)/2$ and $l_{n+1}=l_n(1-\lambda)/2$ by our previous derivation.
	Solving this recursion yields $l_n=[(1-\lambda)/2]^n$.
	
	Let $\delta>0$ be given.
	Choose $n$ such that $l_{n+1}\leq\delta< l_n$.
	Since $\delta \geq l_{n+1}$, we need no more $2^{n+1}$ sets of diameter $\delta$ to cover $C$ because there are exactly $2^{n+1}$ intervals of length $l_{n+1}$ in $C_{n+1}$.
	Furthermore, since $\delta < l_n$, we need at least $2^n$ sets of diameter $\delta$ to cover $C$ becayse there are exactly $2^n$ intervals of length $l_n$ is $C_n$.
	Thus yielding the inequality
	\[
	2^n \leq N^D_\delta(C) \leq 2^{n+1}.
	\]
	Taking logs and doing some manipulation yields
	\[
		\frac{n\ln2}{(n+1)\ln \paren{\frac{2}{1-\lambda}}}\leq\frac{\ln N^D_\delta(C)}{-\ln\delta}\leq\frac{(n+1)\ln2}{n\ln \paren{\frac{2}{1-\lambda}}}.
	\]
	Using L'H\^{o}pital's rule yields
	\[
		\dim_B(C)=\frac{\ln2}{\ln\paren{\frac{2}{1-\lambda}}}.
	\]
\end{proof}

\begin{example}
	Let $F$ be the set containing all numbers in $[0,1]$ that do not have a 5 in their decimal expansion.
	Then,
	\[
		\dim_B(F)=\frac{\ln9}{\ln10}.
	\]
\end{example}

\begin{proof}
	We perform a similar construction to the middle-$\lambda$ Cantor set.
	Define $F_n=[0,10^{-n}]$ where $n$ is a non-negative integer.
	In our construction, we will spilt $F_n$ into ten intervals of length $10^{n+1}$ and remove the interval $(5\cdot10^{n+1},6\cdot10^{n+1})$, removing any number with a 5 in the $n+1$st decimal place from $F_n$.
	We now have nine subintervals left, all of which are copies of $F_{n+1}$.
	This construction means that we have $9^n$ copies of $F_n$ for each $n$.
	
	Let $\delta >0$ be given.
	Choose $n$ such that $10^{-(n+1)}\leq \delta<10^{-n}$.
	Since $\delta\geq 10^{-(n+1)}$, we need no more than $9^{n+1}$ sets of diameter $\delta$ to cover $F$.
	Since $\delta<10^{-n}$, we need at least $9^n$ sets of diameter $\delta$ to cover $F$.
	Thus we have the following inequality
	\[
		9^n \leq N^D_\delta(F)\leq 9^{n+1}.
	\]
	By taking logs and manipulating the inequality we get,
	\[
		\frac{n\ln 9}{(n+1)\ln 10}\leq\frac{\ln N^D_\delta(F)}{-\ln\delta}\leq \frac{(n+1)\ln9}{n\ln 10}.
	\]
	Since both the far left and far right tend to $\frac{\ln9}{\ln10}$, by Squeeze theorem, we have
	\[
		\dim_B(F)=\frac{\ln9}{\ln10}.
	\]
\end{proof}

\begin{example}
	The Cantor Dust has $\dim_B=1$.
\end{example}

\begin{proof}
	To construct the Cantor Dust, we consider the unit square, call this $C_0$.
	Then subdivide $C_0$ into 16 more squares and keep precisely 4 of them, call them copies of $C_1$.
	We iterate this process by splitting $C_n$ into 16 squares and keeping 4 of them, while calling each one $C_{n+1}$.
	We should note that the side length of all of the $4^n$ copies of $C_n$ is $(1/4)^n$.

	Let $\delta>0$ be given.
	We begin by selecting an $n$ such that $(1/4)^{n+1}\leq \delta< (1/4)^n$.
	Thus we need no more than $4^{n+1}$ squares of side length $\delta$ and no fewer than $4^n$ squares of side length $\delta$ to cover $C$.
	However, since this is a cover of squares of side length $\delta$, we have a cover using balls of diameter $\sqrt{2}\delta$, by Pythagorean theorem.
	Thus yielding the following inequality
	\[
		4^{n+1}\leq N^D_{\sqrt{2}\delta}\leq 4^n.
	\]
	After taking logs and some algebraic manipulation, we have
	\[
		\frac{(n+1)\ln4}{n\ln4}\leq \frac{\ln N^D_{\sqrt{2}\delta}}{-\ln\delta}\leq \frac{n\ln4}{(n+1)\ln4}.
	\]
	After we take the limit of both sides we get, $\dim_B(C)=1$.
\end{proof}

However, while the Box-counting dimension is nice for our Cantor-like sets, it has a certain undesirable property, that is countable sets can have non-zero dimension.

\begin{lemma}\label{lems}
	Let $\set{x_n}_{n=1}^\infty$, be a monotone decreasing sequence in $[0,1]$ that tends to 0 and let $0 < \delta < 1/2$.
	Let $S=\set{x_n}_{n=1}^\infty\cup\set{0}$.
	Define
	\begin{align*}
		l_S(n)&:= x_n-x_{n+1}\\
		r_S(n)&:= x_{n-1}-x_n.
	\end{align*}
	Let $k\in\NN$ such that $l_S(k)\leq \delta < r_S(k)$, then
	\[
		k \leq N^D_\delta(S)\leq 2k.
	\]
	and
	\[
		\frac{\ln k}{-\ln r_S(k)}\leq \frac{\ln N^C_\delta}{-\ln\delta}\leq\frac{\ln (2k)}{-\ln l_S(k)}.
	\]
\end{lemma}
\begin{proof}
	If we cover using sets of diameter $r_S(k)$, then we can fit at most one point from $\set{x_j}_{j=1}^k$ in each set.
	Thus implying that $k\leq N^D_\delta(F)$.
	If we cover using sets of diameter $l_S(k)$, then we can cover $[0,x_k]$ with at most, $k+1$ sets and $\set{x_j}_{j=1}^{k-1}$ with $k-1$ sets.
	Thus implying that $k\leq N^D_\delta(F)\leq 2k$, and the logarithm inequality directly follows.
\end{proof}

\begin{example}
	Let $j\in\RR^+$, then the box-counting dimension of $S_j=\set{n^{-j}}_{n=1}^\infty\cup\set{0}$ is
	\[
		\dim_B(S_j)=\frac{1}{j+1}.
	\]
\end{example}
\begin{proof}
	From the construction of $S_j$ we know that
	\[
		l_{S_j}(k) = \frac{1}{k^j}-\frac{1}{(k+1)^j} = \frac{(k+1)^j-k^j}{k^j(k+1)^j}
	\]
	and
	\[
		r_{S_j}(k) = \frac{1}{(k-1)^j}-\frac{1}{k^j} = \frac{k^j-(k-1)^j}{k^j(k-1)^j}.
	\]
	By Lemma\autoref{lems}, we know that
	\[
		\frac{\ln k}{-\ln r_{S_j}(k)}\leq \frac{\ln N^C_\delta}{-\ln\delta}\leq\frac{\ln (2k)}{-\ln l_{S_j}(k)}.
	\]
	Thus, the box-counting dimension is in between the limits of the far left and far right of this inequality.
	Consider $\ln k/(-\ln r_{S_j}(k))$.
	\[
		\frac{\ln k}{-\ln r_{S_j}(k)}=\frac{\ln k}{-\ln (k^j-(k-1)^j)+j\ln k +j\ln(k+1)}
	\]
	Consider $n^j-(n-1)^j$.
	\[
		n^j-(n-1)^j=n^{j-1}\paren{ n - (n-1)\paren{\frac{n-1}{n}}^{j-1} }=n^{j-1}\paren{ n - (n-1)\paren{1-\frac{1}{n}}^{j-1} }
	\]
	Thus,
	\[
		\frac{\ln k}{-\ln r_{S_j}(k)}=\frac{\ln k}{-\ln(k-(k-1)(1-1/k)^{j-1})+\ln k +j\ln(k+1)}
	\]
	which tends to $1/(j+1)$.
	Consider $\ln 2k/(-\ln l_{S_j}(k))$.
	\[
		\frac{\ln 2k}{-\ln l_{S_j}(k)}=\frac{\ln 2k}{-\ln ((k+1)^j-k^j)+j\ln k +j\ln(k-1)}
	\]
	Consider $(n+1)^j-n^j$.
	\[
		(n+1)^j-n^j=n^{j-1}\paren{ (n+1)\paren{\frac{n+1}{n}}^{j-1} - n }=n^{j-1}\paren{ (n+1)\paren{1+\frac{1}{n}}^{j-1} - n }
	\]
	Thus,
	\[
		\frac{\ln 2 +\ln k}{-\ln l_{S_j}(k)}=\frac{\ln 2+\ln k}{-\ln((k+1)(1+1/k)^{j-1}-k)+\ln k +j\ln(k+1)}
	\]
	which tends to $1/(j+1)$ and by Squeeze theorem, $\dim_B(S_j)$ is $1/(j+1)$.
\end{proof}

\begin{example}
	Let $S=\set{(n!)^{-1}}_{n=1}^\infty\cup\set{0}$, then
	\[
		\dim_B(S)=0.
	\]
\end{example}
\begin{proof}
	We begin by computing $l_S(n)$ and $r_S(n)$.
	\begin{align*}
		l_S(n) &= \frac{n}{(n+1)!}\\
		r_S(n) &= \frac{n-1}{n!}
	\end{align*}
	We then invoke Lemma\autoref{lems} to get
	\[
		\frac{\ln k}{-\ln (k-1)+\ln(k!)} \leq \frac{\ln N^C_\delta(S)}{-\ln\delta} \leq \frac{\ln 2+\ln k}{-\ln k +\ln ((k+1)!)}.
	\]
	Since the limit of both the far right and the far left of the previous inequality is zero, $\dim_B(S)=0$.
\end{proof}

