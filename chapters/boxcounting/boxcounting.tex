
\chapter{Box-counting Dimension}

\section{Introduction}

\section{Covers}

Let $F$ be a subset of $\RR^n$.
We define the following covers:
\begin{align*}
	\mcb_\delta(F)&:=\set{B(x_1,\delta),B(x_2,\delta),\ldots,B(x_{n_1},\delta)| F\subseteq \cup_{j=1}^{n_1} B(x_j,\delta)}\\
	\mcc_\delta(F)&:=\set{\bar{B}(x_1,\delta),\bar{B}(x_1,\delta),\ldots,\bar{B}(x_{n_2},\delta)| F\subseteq \cup_{j=1}^{n_2} \bar{B}(x_j,\delta)}\\
	\mcd_\delta(F)&:=\set{D_1,D_2,\ldots, D_{n_3}| F\subseteq \cup_{j=1}^{n_3} D_j \,\,\wedge\,\, \diam(D_j)\leq\delta\,\, \forall\,\, 1\leq j\leq n_3}.
\end{align*}
We now define the following counts
\begin{align*}
	N^B_\delta(F)&:=\min\limits_{\mcb_\delta(F)}|\mcb_\delta(F)|\\
	N^C_\delta(F)&:=\min\limits_{\mcc_\delta(F)}|\mcc_\delta(F)|\\
	N^D_\delta(F)&:=\min\limits_{\mcd_\delta(F)}|\mcd_\delta(F)|.
\end{align*}

\begin{thm}
	\[
		N^B_\delta(F)=N^D_\delta(F)
	\]
\end{thm}
\begin{proof}
	The proof is trivial and left as an excercise.
\end{proof}

\section{Box-counting Dimension}

\begin{definition}
	We say the \textit{lower box-counting dimension} of $F\subset\RR^n$ is
	\[
		\diml_B(F):=\liminf\limits_{\delta\rightarrow 0} \frac{\log N^B_\delta(F)}{-\log\delta}
	\]
	and the \textit{upper box-counting dimension} is
	\[
		\dimu_B(F):=\limsup\limits_{\delta\rightarrow 0} \frac{\log N^B_\delta(F)}{-\log\delta}.
	\]
	If $\diml_B(F)=\dimu_B(F)$, then we say the \textit{box-counting dimension} of $F$ is
	\[
		\dim_B(F):=\lim\limits_{\delta\rightarrow 0} \frac{\log N^B_\delta(F)}{-\log\delta}.
	\]
\end{definition}

\begin{thm}
	$N^C_\delta(F)$ yields the same box-counting dimension as $N^D_\delta(F)$.
\end{thm}
\begin{proof}
	The details are trivial and left as an exercise.
\end{proof}

\section{Examples}

\begin{lemma}
	Let $\set{x_n}_{n=1}^\infty$, be a monotone decreasing sequence in $[0,1]$ that tends to 0 and let $0 < \delta < 1/2$.
	Let $S=\set{x_n}_{n=1}^\infty\cup\set{0}$.
	Define
	\begin{align*}
		l_S(n)&:= x_n-x_{n+1}\\
		r_S(n)&:= x_{n-1}-x_n.
	\end{align*}
	Let $k\in\NN$ such that $l_S(k)\leq \delta < r_S(k)$, then
	\[
		k \leq N^C_\delta(S)\leq 2k.
	\]
\end{lemma}
\begin{proof}
	The proof is obvious and thus is omitted here.
\end{proof}

\begin{example}
	Let $j\in\QQ^+$, then the box-counting dimension of $S_j=\set{n^{-j}}_{n=1}^\infty\cup\set{0}$ is
	\[
		\dim_B(S_j)=\frac{1}{j+1}.
	\]
\end{example}
\begin{proof}
	Proceed via proof by beer.
\end{proof}

\begin{example}
	Let $S=\set{(n!)^{-1}}_{n=1}^\infty\cup\set{0}$, then
	\[
		\dim_B(S)=0.
	\]
\end{example}
\begin{proof}
	Assume the result.
\end{proof}

\begin{example}
	Let $C$ denote the middle-third Cantor set.
	\[
		\dim_B(C)=\frac{\ln2}{\ln3}
	\]
\end{example}

